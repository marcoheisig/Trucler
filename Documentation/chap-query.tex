\chapter{Querying the environment}

\label{chap-environment-querying}

In this chapter, we describe classes and functions that are used by
the compiler to query the environment concerning information about
program elements that the compiler needs in order to determine how to
process those program elements. 

When the compiler calls a generic query function, it passes two or
three arguments, depending on the function it calls.  The first
argument is called the \texttt{client}.  \sysname{} does not
specialize on this argument, but client code should define a standard
class and pass an instance of that class for this argument.  Client
code can then define auxiliary methods that specialize to this class
on the query generic functions.  The second argument is the
environment concerned by the query.  Client code must supply methods
on these functions, specialized to its particular representation of
environments.

These methods should return instances of the classes described in this
chapter.  Any such instance contains all available information about
some program element in that particular environment.  This information
must typically be assembled from different parts of the environment.
For that reason, client code typically creates a new instance whenever
a query function is called, rather than attempting to store such
instances in the environment.  If any of these client-supplied methods
fails to accomplish its task, it should return \texttt{nil}.

Client code is free to define subclasses of the classes described
here, for instance in order to represent implementation-specific
information about the program elements.  Client code would then
typically also provide auxiliary methods or overriding primary methods
on the compilation functions that handle these classes.

\section{Mixin classes}

\subsection{\texttt{name-mixin}}

\Defclass {name-mixin}

This class is a superclass of query classes that require a name to
identify the information supplied by the class instances.

\Definitarg {:name}

\Defmethod {name} {(info {\tt name-mixin})}

Given an instance of the class \texttt{name-mixin}, this method
returns the name information, as supplied by the initarg
\texttt{:name}.

\subsection{\texttt{identity-mixin}}

\Defclass {identity-mixin}

This class is a superclass of query classes that require some kind of
identity to distinguish instances of the query class that have the
same name.

\Definitarg {:identity}

\Defmethod {identity} {(info {\tt identity-mixin})}

Given an instance of the class \texttt{identity-mixin}, this method
returns the identity information, as supplied by the initarg
\texttt{:identity}.

\subsection{\texttt{type-mixin}}

\Defclass {type-mixin}

This class is a superclass of query classes that provide information
of entities that can have a type.

\Definitarg {:type}

If this initarg is not supplied, it defaults to \texttt{t}.

\Defmethod {type} {(info {\tt type-mixin})}

Given an instance of the class \texttt{type-mixin}, this method
returns the type information, as supplied by the initarg
\texttt{:type}.

\subsection{\texttt{ignore-mixin}}

\Defclass {ignore-mixin}

This class is a superclass of query classes that provide information
of entities that can be declared \texttt{ignore} or \texttt{ignorable}.

\Definitarg {:ignore}

\Defmethod {ignore} {(info {\tt ignore-mixin})}

Given an instance of the class \texttt{ignore-mixin}, this method
returns the ignore information, as supplied by the initarg
\texttt{:ignore}.

\subsection{\texttt{dynamic-extent-mixin}}

\Defclass {dynamic-extent-mixin}

This class is a superclass of query classes that provide information
of entities that can be declared \texttt{dynamic-extent}.

\Definitarg {:dynamic-extent}

\Defmethod {dynamic-extent} {(info {\tt dynamic-extent-mixin})}

Given an instance of the class \texttt{dynamic-extent-mixin}, this method
returns the dynamic-extent information, as supplied by the initarg
\texttt{:dynamic-extent}.

\subsection{\texttt{expansion-mixin}}

This class is a superclass of query classes that provide information
of entities that can have an expansion.  In particular, it is a
superclass of the abstract class \texttt{symbol-macro-info}.

\Definitarg {:expansion}

\Defmethod {expansion} {(info {\tt expansion-mixin})}

Given an instance of the class \texttt{expansion-mixin}, this method
returns the expansion information, as supplied by the initarg
\texttt{:expansion}.

\subsection{\texttt{expander-mixin}}

This class is a superclass of query classes that provide information
of entities that can have an expander function.  In particular, it is
a superclass of the abstract class \texttt{macro-info}.

\Definitarg {:expander}

\Defmethod {expander} {(info {\tt expander-mixin})}

Given an instance of the class \texttt{expander-mixin}, this method
returns the expander information, as supplied by the initarg
\texttt{:expander}.

\subsection{\texttt{class-name-mixin}}

This class is a superclass of query classes that provide information
of entities that can have a class-name.  In particular, it is a
superclass of the class \texttt{global-function-info}.

\Definitarg {:class-name}

\Defmethod {class-name} {(info {\tt class-name-mixin})}

Given an instance of the class \texttt{class-name-mixin}, this method
returns the class-name information, as supplied by the initarg
\texttt{:class-name}.

\subsection{\texttt{inline-mixin}}

This class is a superclass of query classes that provide information
of entities that can have inline information.  In particular, it is a
superclass of the class \texttt{authentic-function-info}.

\Definitarg {:inline}

Possible values for this initarg are \texttt{nil}, \texttt{inline},
and \texttt{notinline}, all symbols in the \texttt{common-lisp}
package.  The value \texttt{nil} means that no inline information has
been provided, and this is the default value if the initarg is omitted.

\Defmethod {inline} {(info {\tt inline-mixin})}

Given an instance of the class \texttt{inline-mixin}, this method
returns the inline information, as supplied by the initarg
\texttt{:inline}.

\subsection{\texttt{method-class-name-mixin}}

This class is a superclass of query classes that provide information
of entities that can have method-class-name information.  In
particular, it is a superclass of the class
\texttt{generic-function-info}.

\Definitarg {:method-class-name}

The value of this initarg is a symbol naming a class to be used for
methods.  If this initarg is not given, it defaults to the symbol
\texttt{standard-method}.

\Defmethod {method-class-name} {(info {\tt method-class-name-mixin})}

Given an instance of the class \texttt{method-class-name-mixin}, this
method returns the method-class-name information, as supplied by the
initarg \texttt{:method-class-name}.

\subsection{\texttt{speed-mixin}}

This class is a superclass of query classes that provide information
of entities that can have speed information.  In particular, it is a
superclass of the class \texttt{optimize-info}.

\Definitarg {:speed}

The value of this initarg must be an integer between $0$ and $3$
inclusive.

\Defmethod {speed} {(info {\tt speed-mixin})}

Given an instance of the class \texttt{speed-mixin}, this method
returns the compilation-speed information, as supplied by the initarg
\texttt{:speed}.

\subsection{\texttt{compilation-speed-mixin}}

This class is a superclass of query classes that provide information
of entities that can have compilation-speed information.  In particular, it is a
superclass of the class \texttt{optimize-info}.

\Definitarg {:compilation-speed}

The value of this initarg must be an integer between $0$ and $3$
inclusive.

\Defmethod {compilation-speed} {(info {\tt compilation-speed-mixin})}

Given an instance of the class \texttt{compilation-speed-mixin}, this method
returns the compilation-compilation-speed information, as supplied by the initarg
\texttt{:compilation-speed}.

\subsection{\texttt{debug-mixin}}

This class is a superclass of query classes that provide information
of entities that can have debug information.  In particular, it is a
superclass of the class \texttt{optimize-info}.

\Definitarg {:debug}

The value of this initarg must be an integer between $0$ and $3$
inclusive.

\Defmethod {debug} {(info {\tt debug-mixin})}

Given an instance of the class \texttt{debug-mixin}, this method
returns the compilation-debug information, as supplied by the initarg
\texttt{:debug}.

\subsection{\texttt{space-mixin}}

This class is a superclass of query classes that provide information
of entities that can have space information.  In particular, it is a
superclass of the class \texttt{optimize-info}.

\Definitarg {:space}

The value of this initarg must be an integer between $0$ and $3$
inclusive.

\Defmethod {space} {(info {\tt space-mixin})}

Given an instance of the class \texttt{space-mixin}, this method
returns the compilation-space information, as supplied by the initarg
\texttt{:space}.

\subsection{\texttt{safety-mixin}}

This class is a superclass of query classes that provide information
of entities that can have safety information.  In particular, it is a
superclass of the class \texttt{optimize-info}.

\Definitarg {:safety}

The value of this initarg must be an integer between $0$ and $3$
inclusive.

\Defmethod {safety} {(info {\tt safety-mixin})}

Given an instance of the class \texttt{safety-mixin}, this method
returns the compilation-safety information, as supplied by the initarg
\texttt{:safety}.

\subsection{\texttt{superclass-names-mixin}}

This class is a superclass of query classes that provide information
of entities that can have superclass-names information.  In
particular, it is a superclass of the class \texttt{class-info}.

\Definitarg {:superclass-names}

The value of this initarg is a list of symbols naming a classes.  If
this initarg is not given, it defaults the empty list.  Only
explicitly mentioned superclass names should be provided.

\Defmethod {superclass-names} {(info {\tt superclass-names-mixin})}

Given an instance of the class \texttt{superclass-names-mixin}, this
method returns the superclass-names information, as supplied by the
initarg \texttt{:superclass-names}.

\subsection{\texttt{metaclass-name-mixin}}

This class is a superclass of query classes that provide information
of entities that can have metaclass-name information.  In
particular, it is a superclass of the class
\texttt{class-info}.

\Definitarg {:metaclass-name}

The value of this initarg is a symbol naming a class to be used as a
metaclasss.  If this initarg is not given, it defaults to the symbol
\texttt{standard-class}.

\Defmethod {metaclass-name} {(info {\tt metaclass-name-mixin})}

Given an instance of the class \texttt{metaclass-name-mixin}, this
metaclass returns the metaclass-name information, as supplied by the
initarg \texttt{:metaclass-name}.

\section{Abstract query classes}

\Defclass {variable-info}

This abstract class is the superclass of every query class returned by
a call to the generic function \texttt{variable-info}.  It is a
subclass of the class \texttt{name-mixin}.

\Defclass {authentic-variable-info}

This abstract class is a subclass of the classes
\texttt{variable-info} and \texttt{type-mixin}.

It is a superclass of the two instantiable classes
\texttt{lexical-variable-info} and
\texttt{special-variable-info}.

\Defclass {symbol-macro-info}

This abstract class is a subclass of the classes
\texttt{variable-info}, \texttt{type-mixin}, and
\texttt{expansion-mixin}.

It is a superclass of the two instantiable classes
\texttt{local-symbol-macro-info} and
\texttt{global-symbol-macro-info}.

\Defclass {function-info}

This abstract class is the superclass of every query class returned by
a call to the generic function \texttt{function-info}.  It is a
subclass of the class \texttt{name-mixin}.

\Defclass {authentic-function-info}

This abstract class is a subclass of the classes
\texttt{function-info} and \texttt{type-mixin}.

It is a superclass of the two instantiable classes
\texttt{local-function-info} and
\texttt{global-function-info}.

\Defclass {macro-info}

This abstract class is a subclass of the classes
\texttt{function-info} and \texttt{expander-mixin}.

It is a superclass of the two instantiable classes
\texttt{local-macro-info} and
\texttt{global-macro-info}.

\section{Variable information}

\Defgeneric {variable-info} {client environment symbol}

This function is called by the compiler whenever a symbol in a
\emph{variable} position is to be compiled.  If successful, it returns an instance of
one of the classes described below.

\Defcondition {no-variable-info}

This condition is signaled by \sysname{} when a client-supplied method
on the generic function \texttt{variable-info} returns \texttt{nil}.

\Defmethod {name} {(condition {\tt no-variable-info})}

This method returns the name of the variable for which no info was
available. 

\subsection{Lexical variable information}

\Defclass {lexical-variable-info}

This class represents information about lexical variables.  An
instance of this class is returned by a call to \texttt{variable-info}
when it turns out that the symbol passed as an argument refers to a
lexical variable.

This class is a subclass of the classes
\texttt{authentic-variable-info} \texttt{identity-mixin},
\texttt{ignore-mixin}, and \texttt{dynamic-extent-mixin}.

\subsection{Special variable information}

\Defclass {special-variable-info}

This class represents information about special variables.   An
instance of this class is returned by a call to \texttt{variable-info}
when it turns out that the symbol passed as an argument refers to a
special variable.

This class is a subclass of the class \texttt{authentic-variable-info}.

\Definitarg {:global-p}

This initarg is used to supply information on whether the variable is globally or locally special, i.e. whether it is special because of a proclamation. The value of this initarg can be either \texttt{t}, meaning that the variable is globally special, or \texttt{nil}, meaning that it is not.

\Defmethod {global-p} {(info {\tt special-variable-info})}

Given an instance of the class \texttt{special-variable-info}, this method returns the \emph{global-p} information of the special variable as supplised by the initarg \texttt{:global-p}.

\subsection{Constant variable information}

\Defclass {constant-variable-info}

This class represents information about constant variables.   An
instance of this class is returned by a call to \texttt{variable-info}
when it turns out that the symbol passed as an argument refers to a
constant variable.

This class is a subclass of the class \texttt{variable-info}.

\Definitarg {:value}

This initarg supplies the value of the constant variable.  This
initarg must be supplied.

\Defmethod {value} {(info {\tt constant-variable-info})}

Given an instance of the class \texttt{constant-variable-info}, this
method returns the value of the constant variable as supplied by the
initarg \texttt{:value}.

\subsection{Symbol macro information}

\Defclass {global-symbol-macro-info}

This class is a subclass of \texttt{symbol-macro-info}.  It is
returned by a call to \texttt{variable-information} when the name is
defined as a global symbol macro, as defined by
\texttt{define-symbol-macro}.

\Defclass {local-symbol-macro-info}

This class is a subclass of \texttt{symbol-macro-info} and
\texttt{ignore-mixin}.  It is returned by a call to
\texttt{variable-information} when the name is defined as a local
symbol macro, as defined by \texttt{symbol-macrolet}.

\section{Function information}

\Defgeneric {function-info} {client environment function-name}

This function is called by the compiler whenever a symbol in a
\emph{function} position is to be compiled or whenever a function name
is found in a context where it is known to refer to a function.  It
returns an instance of one of the classes described below.

\Defcondition {no-function-info}

This condition is signaled by \sysname{} when a client-supplied method
on the generic function \texttt{function-info} returns \texttt{nil}.

\Defmethod {name} {(condition {\tt no-function-info})}

This method returns the name of the function for which no info was
available. 

\subsection{Local function information}

\Defclass {local-function-info}

This class represents information about local functions introduced by
\texttt{flet} or \texttt{labels}.  An instance of this class is
returned by a call to \texttt{function-info} when it turns out that
the function name passed as an argument refers to a local function. 

\Definitarg {:name}

This initarg supplies the name of the local function.  This initarg
must be supplied.

\Definitarg {:identity}

This initarg is used to supply some kind of implementation-defined 
\emph{identity}.  The implementation can supply any object as the
identity, because it is not interpreted by the compiler.  However, the
\emph{same} identity must be supplied each time for a particular
local function.  This initarg must be supplied. 

\Definitarg {:type}

This initarg is used to supply the \emph{type} of the local function.
The type can be any function type specifier and it may contain
user-defined types.  If this initarg is omitted, it defaults to
\texttt{t}.

\Definitarg {:inline}

This initarg is used to supply \emph{inline} information about the
local function.  The value of this initarg can be either
\texttt{inline} (i.e., the symbol with that name in the
\texttt{common-lisp} package) meaning that the function is declared
\texttt{inline}, \texttt{notinline} (i.e., the symbol with that name
in the \texttt{common-lisp} package) meaning that the function is
declared \texttt{notinline}, or \texttt{nil} meaning that no inline
declaration for this function is in scope.  If this initarg is not
supplied, it defaults to \texttt{nil}.

\Definitarg {:ignore}

This initarg is used to supply \emph{ignore} information about the
local function.  The value of this initarg can be either
\texttt{ignore} (i.e., the symbol with that name in the
\texttt{common-lisp} package) meaning that the function is declared
\texttt{ignore}, \texttt{ignorable} (i.e., the symbol with that name
in the \texttt{common-lisp} package) meaning that the function is
declared \texttt{ignorable}, or \texttt{nil} meaning that no ignore or
ignorable declaration for this function is in scope.  If this initarg
is not supplied, it defaults to \texttt{nil}.

\Definitarg {:dynamic-extent}

This initarg is used to supply \emph{dynamic extent} information about
the local function.  The value of this initarg can be either
\texttt{t}, meaning that the function has been declared
\texttt{dynamic-extent}, or \texttt{nil}, meaning that the function
has not been declared \texttt{dynamic-extent}.  The default value when
this initarg is not supplied is \texttt{nil}. 

\Defmethod {name} {(info {\tt local-function-info})}

Given an instance of the class \texttt{local-function-info}, this
method returns the name of the local function as supplied by the
initarg \texttt{:name}.

\Defmethod {identity} {(info {\tt local-function-info})}

Given an instance of the class \texttt{local-function-info}, this
method returns the identity of the local function as supplied by the
initarg \texttt{:identity}.

\Defmethod {type} {(info {\tt local-function-info})}

Given an instance of the class \texttt{local-function-info}, this
method returns the \emph{type} of the local function as supplied by the
initarg \texttt{:type}.  If that initarg was not supplied, this method
returns \texttt{t}.

\Defmethod {inline} {(info {\tt local-function-info})}

Given an instance of the class \texttt{local-function-info}, this
method returns the \emph{inline} information of the local function as
supplied by the initarg \texttt{:inline}.  If that initarg was not
supplied, this method returns \texttt{nil}.

\Defmethod {ignore} {(info {\tt local-function-info})}

Given an instance of the class \texttt{local-function-info}, this
method returns the \emph{ignore} information of the local function as
supplied by the initarg \texttt{:ignore}.  If that initarg was not
supplied, this method returns \texttt{nil}.

\Defmethod {dynamic-extent} {(info {\tt local-function-info})}

Given an instance of the class \texttt{local-function-info}, this
method returns the \emph{dynamic extent} information of the local
function as supplied by the initarg \texttt{:dynamic-extent}.  If that
initarg was not supplied, this method returns \texttt{nil}.

\subsection{Global function information}

\Defclass {global-function-info}

This class represents information about global functions.  An instance
of this class is returned by a call to \texttt{function-info} when it
turns out that the function name passed as an argument refers to a
global function.

\Definitarg {:name}

This initarg supplies the name of the global function.  This initarg
must be supplied.

\Definitarg {:type}

This initarg is used to supply the \emph{type} of the global function.
The type can be any function type specifier and it may contain
user-defined types.  If this initarg is omitted, it defaults to
\texttt{t}.

\Definitarg {:inline}

This initarg is used to supply \emph{inline} information about the
global function.  The value of this initarg can be either
\texttt{inline} (i.e., the symbol with that name in the
\texttt{common-lisp} package) meaning that the function is declared
\texttt{inline}, \texttt{notinline} (i.e., the symbol with that name
in the \texttt{common-lisp} package) meaning that the function is
declared \texttt{notinline}, or \texttt{nil} meaning that no inline
declaration for this function is in scope.  If this initarg is not
supplied, it defaults to \texttt{nil}.

\Definitarg {:ignore}

This initarg is used to supply \emph{ignore} information about the
global function.  The value of this initarg can be either
\texttt{ignore} (i.e., the symbol with that name in the
\texttt{common-lisp} package) meaning that the function is declared
\texttt{ignore}, \texttt{ignorable} (i.e., the symbol with that name
in the \texttt{common-lisp} package) meaning that the function is
declared \texttt{ignorable}, or \texttt{nil} meaning that no ignore or
ignorable declaration for this function is in scope.  If this initarg
is not supplied, it defaults to \texttt{nil}.

\Definitarg {:compiler-macro}

This initarg is used to supply a \emph{compiler macro function} when
a compiler macro is associated with the global function.  If this
initarg is not given, it defaults to \texttt{nil}, meaning that no
compiler macro is associated with this function. 

\Definitarg {:class-name}

This initarg is used to supply the name of the class of the global
function.  If this initarg is not given, it defaults to
\texttt{function}.

\Defmethod {name} {(info {\tt global-function-info})}

Given an instance of the class \texttt{global-function-info}, this
method returns the name of the global function as supplied by the
initarg \texttt{:name}.

\Defmethod {type} {(info {\tt global-function-info})}

Given an instance of the class \texttt{global-function-info}, this
method returns the \emph{type} of the global function as supplied by the
initarg \texttt{:type}.  If that initarg was not supplied, this method
returns \texttt{t}.

\Defmethod {inline} {(info {\tt global-function-info})}

Given an instance of the class \texttt{global-function-info}, this
method returns the \emph{inline} information of the global function as
supplied by the initarg \texttt{:inline}.  If that initarg was not
supplied, this method returns \texttt{nil}.

\Defmethod {ignore} {(info {\tt global-function-info})}

Given an instance of the class \texttt{global-function-info}, this
method returns the \emph{ignore} information of the global function as
supplied by the initarg \texttt{:ignore}.  If that initarg was not
supplied, this method returns \texttt{nil}.

\Defmethod {inline} {(info {\tt global-function-info})}

Given an instance of the class \texttt{global-function-info}, this
method returns the \emph{inline} information of the global function as
supplied by the initarg \texttt{:inline}. If that initarg was not
supplied, this method returns \texttt{nil}.

\Defmethod compiler-macro {(info {\tt global-function-info})}

Given an instance of the class \texttt{global-function-info}, this
method returns the \emph{compiler macro function} information
associated with the global function, as supplied by the initarg
\texttt{:compiler-macro}.  If that initarg was not supplied, this
method returns \texttt{nil}.

\Defmethod class-name {(info {\tt global-function-info})}

Given an instance of the class \texttt{global-function-info}, this
method returns the \emph{class name} information
associated with the global function, as supplied by the initarg
\texttt{:class-name}.  If that initarg was not supplied, this
method returns \texttt{function}.

\Defclass {generic-function-info}

This class is a subclass of \texttt{global-function-info}

If the \texttt{:class-name} initarg is not given when an instance of
this class is created, it defaults to
\texttt{standard-generic-function}.

\Definitarg {:method-class-name}

This initarg is used to supply the name of the class to be used for
methods on the generic function.  If this initarg is not given, it
defaults to \texttt{standard-method}.

\Defmethod method-class-name {(info {\tt generic-function-info})}

Given an instance of the class \texttt{generic-function-info}, this
method returns the \emph{method class name} information
associated with the global function, as supplied by the initarg
\texttt{:method-class-name}.  If that initarg was not supplied, this
method returns \texttt{standard-method}.

\subsection{Local macro information}

\Defclass {local-macro-info}

This class represents information about local macros introduced by
\texttt{macrolet}.  An instance of this class is returned by a call to
\texttt{function-info} when it turns out that the function name passed
as an argument refers to a local macro.

\Definitarg {:name}

This initarg supplies the name of the local macro.  This initarg
must be supplied.

\Definitarg {:expander}

This initarg is used to supply the macro function used to expand macro
forms that use this macro.  This initarg must be supplied. 

\Defmethod {name} {(info {\tt local-macro-info})}

Given an instance of the class \texttt{local-macro-info}, this
method returns the name of the local macro as supplied by the
initarg \texttt{:name}.

\Defmethod {expander} {(info {\tt local-macro-info})}

Given an instance of the class \texttt{local-macro-info}, this
method returns the expander of the local macro as supplied by the
initarg \texttt{:expander}.

\subsection{Global macro information}

\Defclass {global-macro-info}

This class represents information about global macros introduced by
\texttt{macrolet}.  An instance of this class is returned by a call to
\texttt{function-info} when it turns out that the function name passed
as an argument refers to a global macro.

\Definitarg {:name}

This initarg supplies the name of the global macro.  This initarg
must be supplied.

\Definitarg {:expander}

This initarg is used to supply the macro function used to expand macro
forms that use this macro.  This initarg must be supplied. 

\Definitarg {:compiler-macro}

This initarg is used to supply a \emph{compiler macro function} when
a compiler macro is associated with the global macro.  If this
initarg is not given, it defaults to \texttt{nil}, meaning that no
compiler macro is associated with this macro. 

\Defmethod {name} {(info {\tt global-macro-info})}

Given an instance of the class \texttt{global-macro-info}, this
method returns the name of the global macro as supplied by the
initarg \texttt{:name}.

\Defmethod {expander} {(info {\tt global-macro-info})}

Given an instance of the class \texttt{global-macro-info}, this
method returns the expander of the global macro as supplied by the
initarg \texttt{:expander}.

\Defmethod compiler-macro {(info {\tt global-macro-info})}

Given an instance of the class \texttt{global-macro-info}, this
method returns the \emph{compiler macro function} information
associated with the global macro, as supplied by the initarg
\texttt{:compiler-macro}.  If that initarg was not supplied, this
method returns \texttt{nil}.

\subsection{Special operator information}

\Defclass {special-operator-info}

This class represents information about special operators.  An
instance of this class is returned by a call to \texttt{function-info}
when it turns out that the function name passed as an argument refers
to a specialoperator.

This class is a subclass of the class \texttt{function-info}.

\section{Block information}

\Defgeneric {block-info} {client environment symbol}

\Defcondition {no-block-info}

This condition is signaled by \sysname{} when a client-supplied method
on the generic function \texttt{block-info} returns \texttt{nil}.

\Defmethod {name} {(condition {\tt no-block-info})}

This method returns the name of the block for which no info was
available. 

\Defclass {block-info}

This class represents information about blocks introduced by
\texttt{block}.  An instance of this class is returned by a call to
\texttt{block-info} when the symbol passed as an argument refers to a
block.

This class is a subclass of the classes \texttt{name-mixin} and
\texttt{identity-mixin}.

\section{Tag information}

\Defgeneric {tag-info} {client environment tag}

\Defcondition {no-tag-info}

This condition is signaled by \sysname{} when a client-supplied method
on the generic function \texttt{tag-info} returns \texttt{nil}.

\Defmethod {name} {(condition {\tt no-tag-info})}

This method returns the name of the tag for which no info was
available. 

\Defclass {tag-info}

This class represents information about tags introduced by
\texttt{tagbody}.  An instance of this class is returned by a call to
\texttt{tag-info} when the name (which must be a symbol or an integer)
passed as an argument refers to a tag.

This class is a subclass of the classes \texttt{name-mixin} and
\texttt{identity-mixin}.

\section{Class information}

\Defgeneric {class-info} {client environment class-name}

\Defcondition {no-class-info}

This condition is signaled by \sysname{} when a client-supplied method
on the generic function \texttt{class-info} returns \texttt{nil}.

\Defmethod {name} {(condition {\tt no-class-info})}

This method returns the name of the class for which no info was
available. 

\Defclass {class-info}

This class represents information about a class introduced by
\texttt{defclass}.  An instance of this class is returned by a call to
\texttt{class-info} when the name (which must be a symbol)
passed as an argument refers to a class.

This class is a subclass of the classes \texttt{name-mixin} and
\texttt{identity-mixin}.

\Definitarg {:superclass-names}

This initarg supplies list of names of superclasses that was given in
the \texttt{defclass} form.

\Definitarg {:metaclass-name}

This initarg supplies the name of the metaclass that was given in the
\texttt{defclass} form.  If this initarg is not given, it defaults to
the symbol \texttt{standard-class}.

\Defmethod {superclass-names} {(info {\tt class-info})}

Given an instance of the class \texttt{class-info}, this
method returns the list of superclass names as supplied by the
initarg \texttt{:superclass-names}.

\Defmethod {metaclass-name} {(info {\tt class-info})}

Given an instance of the class \texttt{class-info}, this
method returns the name of the metaclass as supplied by the
initarg \texttt{:metaclass-name}.

\section{Optimization information}

\Defgeneric {optimize-info} {client environment}

Client-supplied methods on this function must always return a valid
instance of the class \texttt{optimize-info}. Optimize info includes
information about optimize declarations.

\Defclass {optimize-info}

\Defmethod {optimize} {(info {\tt optimize-info})}

Given an instance of the class \texttt{optimize-info}, this method 
returns the optimize information as supplied by the initarg
\texttt{:optimize}.

\section{Type information}

\Defgeneric {type-expand} {client environment type}

This generic function is called with some arbitrary type specifier,
and the return value is an equivalent type specifier that does not
contain any user-defined types introduced by \texttt{deftype}. 
