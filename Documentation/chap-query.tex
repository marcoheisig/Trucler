\chapter{Querying the environment}

\label{chap-environment-querying}

In this chapter, we describe classes and functions that are used by
the compiler to query the environment concerning information about
program elements that the compiler needs in order to determine how to
process those program elements. 

When the compiler calls a generic query function, it passes the
environment as the first argument.  Client code must supply methods
on these functions, specialized to its particular representation of
environments. 

These methods should return instances of the classes described in this
chapter.  Any such instance contains all available information about
some program element in that particular environment.  This information
must typically be assembled from different parts of the environment.
For that reason, client code typically creates a new instance whenever
a query function is called, rather than attempting to store such
instances in the environment.  If any of these client-supplied methods
fails to accomplish its task, it should return \texttt{nil}.

Client code is free to define subclasses of the classes described
here, for instance in order to represent implementation-specific
information about the program elements.  Client code would then
typically also provide auxiliary methods or overriding primary methods
on the compilation functions that handle these classes.

\section{Variable information}

\Defgeneric {variable-info} {environment symbol}

This function is called by the compiler whenever a symbol in a
\emph{variable} position is to be compiled.  If successful, it returns an instance of
one of the classes described below.

\Defcondition {no-variable-info}

This condition is signaled by \sysname{} when a client-supplied method
on the generic function \texttt{variable-info} returns \texttt{nil}.

\Defmethod {name} {(condition {\tt no-variable-info})}

This method returns the name of the variable for which no info was
available. 

\subsection{Lexical variable information}

\Defclass {lexical-variable-info}

This class represents information about lexical variables.  An
instance of this class is returned by a call to \texttt{variable-info}
when it turns out that the symbol passed as an argument refers to a
lexical variable.

\Definitarg {:name}

This initarg supplies the name of the lexical variable.  This initarg
must be supplied. 

\Definitarg {:identity}

This initarg is used to supply some kind of implementation-defined 
\emph{identity}.  The implementation can supply any object as the
identity, because it is not interpreted by the compiler.  However, the
\emph{same} identity must be supplied each time for a particular
lexical variable.  This initarg must be supplied. 

\Definitarg {:type}

This initarg is used to supply the \emph{type} of the lexical
variable.  The type can be any type specifier and it may contain
user-defined types.  If this initarg is omitted, it defaults to
\texttt{t}. 

\Definitarg {:ignore}

This initarg is used to supply \emph{ignore} information about the
lexical variable.  The value of this initarg can be either
\texttt{ignore} (i.e., the symbol with that name in the
\texttt{common-lisp} package) meaning that the variable is declared
\texttt{ignore}, \texttt{ignorable} (i.e., the symbol with that name
in the \texttt{common-lisp} package) meaning that the variable is
declared \texttt{ignorable}, or \texttt{nil} meaning that no ignore or
ignorable declaration for this variable is in scope.  If this initarg
is not supplied, it defaults to \texttt{nil}.

\Definitarg {:dynamic-extent}

This initarg is used to supply \emph{dynamic extent} information about
the lexical variable.  The value of this initarg can be either
\texttt{t}, meaning that the variable has been declared
\texttt{dynamic-extent}, or \texttt{nil}, meaning that the variable
has not been declared \texttt{dynamic-extent}.  The default value when
this initarg is not supplied is \texttt{nil}. 

\Defmethod {name} {(info {\tt lexical-variable-info})}

Given an instance of the class \texttt{lexical-variable-info}, this
method returns the name of the lexical variable as supplied by the
initarg \texttt{:name}.

\Defmethod {identity} {(info {\tt lexical-variable-info})}

Given an instance of the class \texttt{lexical-variable-info}, this
method returns the identity of the lexical variable as supplied by the
initarg \texttt{:identity}.

\Defmethod {type} {(info {\tt lexical-variable-info})}

Given an instance of the class \texttt{lexical-variable-info}, this
method returns the \emph{type} of the variable as supplied by the
initarg \texttt{:type}.  If that initarg was not supplied, this method
returns \texttt{t}.

\Defmethod {ignore} {(info {\tt lexical-variable-info})}

Given an instance of the class \texttt{lexical-variable-info}, this
method returns the \emph{ignore} information of the lexical variable as
supplied by the initarg \texttt{:ignore}.  If that initarg was not
supplied, this method returns \texttt{nil}.

\Defmethod {dynamic-extent} {(info {\tt lexical-variable-info})}

Given an instance of the class \texttt{lexical-variable-info}, this
method returns the \emph{dynamic extent} information of the lexical
variable as supplied by the initarg \texttt{:dynamic-extent}.  If that
initarg was not supplied, this method returns \texttt{nil}.

\subsection{Special variable information}

\Defclass {special-variable-info}

This class represents information about special variables.   An
instance of this class is returned by a call to \texttt{variable-info}
when it turns out that the symbol passed as an argument refers to a
special variable.


\Definitarg {:name}

This initarg supplies the name of the special variable.  This initarg
must be supplied. 

\Definitarg {:type}

This initarg is used to supply the \emph{type} of the special
variable.  The type can be any type specifier and it may contain
user-defined types.  If this initarg is omitted, it defaults to
\texttt{t}. 

\Definitarg {:ignore}

This initarg is used to supply \emph{ignore} information about the
special variable.  The value of this initarg can be either
\texttt{ignore} (i.e., the symbol with that name in the
\texttt{common-lisp} package) meaning that the variable is declared
\texttt{ignore}, \texttt{ignorable} (i.e., the symbol with that name
in the \texttt{common-lisp} package) meaning that the variable is
declared \texttt{ignorable}, or \texttt{nil} meaning that no ignore or
ignorable declaration for this variable is in scope.  If this initarg
is not supplied, it defaults to \texttt{nil}.

\Definitarg {:global-p}

This initarg is used to supply information on whether the variable is globally or locally special, i.e. whether it is special because of a proclamation. The value of this initarg can be either \texttt{t}, meaning that the variable is globally special, or \texttt{nil}, meaning that it is not.

\Defmethod {name} {(info {\tt special-variable-info})}

Given an instance of the class \texttt{special-variable-info}, this
method returns the name of the special variable as supplied by the
initarg \texttt{:name}.

\Defmethod {type} {(info {\tt special-variable-info})}

Given an instance of the class \texttt{special-variable-info}, this
method returns the \emph{type} of the variable as supplied by the
initarg \texttt{:type}.  If that initarg was not supplied, this method
returns \texttt{t}.

\Defmethod {ignore} {(info {\tt special-variable-info})}

Given an instance of the class \texttt{special-variable-info}, this
method returns the \emph{ignore} information of the special variable as
supplied by the initarg \texttt{:ignore}.  If that initarg was not
supplied, this method returns \texttt{nil}.

\Defmethod {global-p} {(info {\tt special-variable-info})}

Given an instance of the class \texttt{special-variable-info}, this method returns the \emph{global-p} information of the special variable as supplised by the initarg \texttt{:global-p}.

\subsection{Constant variable information}

\Defclass {constant-variable-info}

This class represents information about constant variables.   An
instance of this class is returned by a call to \texttt{variable-info}
when it turns out that the symbol passed as an argument refers to a
constant variable.

\Definitarg {:name}

This initarg supplies the name of the constant variable.  This initarg
must be supplied. 

\Definitarg {:value}

This initarg supplies the value of the constant variable.  This
initarg must be supplied.

\Defmethod {name} {(info {\tt constant-variable-info})}

Given an instance of the class \texttt{constant-variable-info}, this
method returns the name of the constant variable as supplied by the
initarg \texttt{:name}.

\Defmethod {value} {(info {\tt constant-variable-info})}

Given an instance of the class \texttt{constant-variable-info}, this
method returns the value of the constant variable as supplied by the
initarg \texttt{:value}.

\subsection{Symbol macro information}

\Defclass {symbol-macro-info}

This class represents information about symbol macros.  An
instance of this class is returned by a call to \texttt{variable-info}
when it turns out that the symbol passed as an argument refers to a
symbol macro.

\Definitarg {:name}

This initarg supplies the name of the symbol macro.  This initarg must
be supplied.

\Definitarg {:expansion}

This initarg supplies the expansion of the symbol macro.  This initarg
must be supplied.

\Definitarg {:type}

This initarg is used to supply the \emph{type} of the symbol macro.
The type can be any type specifier and it may contain user-defined
types.  If this initarg is omitted, it defaults to \texttt{t}.

\Defmethod {name} {(info {\tt symbol-macro-info})}

Given an instance of the class \texttt{symbol-macro-info}, this method
returns the name of the symbol macro as supplied by the initarg
\texttt{:name}.

\Defmethod {expansion} {(info {\tt symbol-macro-info})}

Given an instance of the class \texttt{symbol-macro-info}, this method
returns the expansion of the symbol macro as supplied by the initarg
\texttt{:expansion}.

\Defmethod {type} {(info {\tt symbol-macro-info})}

Given an instance of the class \texttt{symbol-macro-info}, this method
returns the \emph{type} of the symbol macro as supplied by the initarg
\texttt{:type}.  If that initarg was not supplied, this method returns
\texttt{t}.

\section{Function information}

\Defgeneric {function-info} {environment function-name}

This function is called by the compiler whenever a symbol in a
\emph{function} position is to be compiled or whenever a function name
is found in a context where it is known to refer to a function.  It
returns an instance of one of the classes described below.

\Defcondition {no-function-info}

This condition is signaled by \sysname{} when a client-supplied method
on the generic function \texttt{function-info} returns \texttt{nil}.

\Defmethod {name} {(condition {\tt no-function-info})}

This method returns the name of the function for which no info was
available. 

\subsection{Local function information}

\Defclass {local-function-info}

This class represents information about local functions introduced by
\texttt{flet} or \texttt{labels}.  An instance of this class is
returned by a call to \texttt{function-info} when it turns out that
the function name passed as an argument refers to a local function. 

\Definitarg {:name}

This initarg supplies the name of the local function.  This initarg
must be supplied.

\Definitarg {:identity}

This initarg is used to supply some kind of implementation-defined 
\emph{identity}.  The implementation can supply any object as the
identity, because it is not interpreted by the compiler.  However, the
\emph{same} identity must be supplied each time for a particular
local function.  This initarg must be supplied. 

\Definitarg {:type}

This initarg is used to supply the \emph{type} of the local function.
The type can be any function type specifier and it may contain
user-defined types.  If this initarg is omitted, it defaults to
\texttt{t}.

\Definitarg {:inline}

This initarg is used to supply \emph{inline} information about the
local function.  The value of this initarg can be either
\texttt{inline} (i.e., the symbol with that name in the
\texttt{common-lisp} package) meaning that the function is declared
\texttt{inline}, \texttt{notinline} (i.e., the symbol with that name
in the \texttt{common-lisp} package) meaning that the function is
declared \texttt{notinline}, or \texttt{nil} meaning that no inline
declaration for this function is in scope.  If this initarg is not
supplied, it defaults to \texttt{nil}.

\Definitarg {:ignore}

This initarg is used to supply \emph{ignore} information about the
local function.  The value of this initarg can be either
\texttt{ignore} (i.e., the symbol with that name in the
\texttt{common-lisp} package) meaning that the function is declared
\texttt{ignore}, \texttt{ignorable} (i.e., the symbol with that name
in the \texttt{common-lisp} package) meaning that the function is
declared \texttt{ignorable}, or \texttt{nil} meaning that no ignore or
ignorable declaration for this function is in scope.  If this initarg
is not supplied, it defaults to \texttt{nil}.

\Definitarg {:ast}

This initarg is used to supply an inline expansion for the local 
function. The value of this initarg can be either a
\texttt{cleavir-ast:function-ast}, or \texttt{nil}, indicating that
no inline expansion is available. If this initarg is not supplied,
it defaults to \texttt{nil}.

\Definitarg {:dynamic-extent}

This initarg is used to supply \emph{dynamic extent} information about
the local function.  The value of this initarg can be either
\texttt{t}, meaning that the function has been declared
\texttt{dynamic-extent}, or \texttt{nil}, meaning that the function
has not been declared \texttt{dynamic-extent}.  The default value when
this initarg is not supplied is \texttt{nil}. 

\Defmethod {name} {(info {\tt local-function-info})}

Given an instance of the class \texttt{local-function-info}, this
method returns the name of the local function as supplied by the
initarg \texttt{:name}.

\Defmethod {identity} {(info {\tt local-function-info})}

Given an instance of the class \texttt{local-function-info}, this
method returns the identity of the local function as supplied by the
initarg \texttt{:identity}.

\Defmethod {type} {(info {\tt local-function-info})}

Given an instance of the class \texttt{local-function-info}, this
method returns the \emph{type} of the local function as supplied by the
initarg \texttt{:type}.  If that initarg was not supplied, this method
returns \texttt{t}.

\Defmethod {inline} {(info {\tt local-function-info})}

Given an instance of the class \texttt{local-function-info}, this
method returns the \emph{inline} information of the local function as
supplied by the initarg \texttt{:inline}.  If that initarg was not
supplied, this method returns \texttt{nil}.

\Defmethod {ignore} {(info {\tt local-function-info})}

Given an instance of the class \texttt{local-function-info}, this
method returns the \emph{ignore} information of the local function as
supplied by the initarg \texttt{:ignore}.  If that initarg was not
supplied, this method returns \texttt{nil}.

\Defmethod {inline} {(info {\tt local-function-info})}

Given an instance of the class \texttt{local-function-info}, this method returns the \emph{ast} information of the local function as supplied by the initarg \texttt{:ast}. If that initarg was not supplied, this method returns \texttt{nil}.

\Defmethod {dynamic-extent} {(info {\tt local-function-info})}

Given an instance of the class \texttt{local-function-info}, this
method returns the \emph{dynamic extent} information of the local
function as supplied by the initarg \texttt{:dynamic-extent}.  If that
initarg was not supplied, this method returns \texttt{nil}.

\subsection{Global function information}

\Defclass {global-function-info}

This class represents information about global functions.  An instance
of this class is returned by a call to \texttt{function-info} when it
turns out that the function name passed as an argument refers to a
global function.

\Definitarg {:name}

This initarg supplies the name of the global function.  This initarg
must be supplied.

\Definitarg {:type}

This initarg is used to supply the \emph{type} of the global function.
The type can be any function type specifier and it may contain
user-defined types.  If this initarg is omitted, it defaults to
\texttt{t}.

\Definitarg {:inline}

This initarg is used to supply \emph{inline} information about the
global function.  The value of this initarg can be either
\texttt{inline} (i.e., the symbol with that name in the
\texttt{common-lisp} package) meaning that the function is declared
\texttt{inline}, \texttt{notinline} (i.e., the symbol with that name
in the \texttt{common-lisp} package) meaning that the function is
declared \texttt{notinline}, or \texttt{nil} meaning that no inline
declaration for this function is in scope.  If this initarg is not
supplied, it defaults to \texttt{nil}.

\Definitarg {:ignore}

This initarg is used to supply \emph{ignore} information about the
global function.  The value of this initarg can be either
\texttt{ignore} (i.e., the symbol with that name in the
\texttt{common-lisp} package) meaning that the function is declared
\texttt{ignore}, \texttt{ignorable} (i.e., the symbol with that name
in the \texttt{common-lisp} package) meaning that the function is
declared \texttt{ignorable}, or \texttt{nil} meaning that no ignore or
ignorable declaration for this function is in scope.  If this initarg
is not supplied, it defaults to \texttt{nil}.

\Definitarg {:ast}

This initarg is used to supply an inline expansion for the global
function. The value of this initarg can be either a
\texttt{cleavir-ast:function-ast}, or \texttt{nil}, indicating that
no inline expansion is available. If this initarg is not supplied,
it defaults to \texttt{nil}.

\Definitarg {:compiler-macro}

This initarg is used to supply a \emph{compiler macro function} when
a compiler macro is associated with the global function.  If this
initarg is not given, it defaults to \texttt{nil}, meaning that no
compiler macro is associated with this function. 

\Defmethod {name} {(info {\tt global-function-info})}

Given an instance of the class \texttt{global-function-info}, this
method returns the name of the global function as supplied by the
initarg \texttt{:name}.

\Defmethod {type} {(info {\tt global-function-info})}

Given an instance of the class \texttt{global-function-info}, this
method returns the \emph{type} of the global function as supplied by the
initarg \texttt{:type}.  If that initarg was not supplied, this method
returns \texttt{t}.

\Defmethod {inline} {(info {\tt global-function-info})}

Given an instance of the class \texttt{global-function-info}, this
method returns the \emph{inline} information of the global function as
supplied by the initarg \texttt{:inline}.  If that initarg was not
supplied, this method returns \texttt{nil}.

\Defmethod {ignore} {(info {\tt global-function-info})}

Given an instance of the class \texttt{global-function-info}, this
method returns the \emph{ignore} information of the global function as
supplied by the initarg \texttt{:ignore}.  If that initarg was not
supplied, this method returns \texttt{nil}.

\Defmethod {inline} {(info {\tt global-function-info})}

Given an instance of the class \texttt{global-function-info}, this method returns the \emph{ast} information of the global function as supplied by the initarg \texttt{:ast}. If that initarg was not supplied, this method returns \texttt{nil}.

\Defmethod compiler-macro {(info {\tt global-function-info})}

Given an instance of the class \texttt{global-function-info}, this
method returns the \emph{compiler macro function} information
associated with the global function, as supplied by the initarg
\texttt{:compiler-macro}.  If that initarg was not supplied, this
method returns \texttt{nil}.

\subsection{Local macro information}

\Defclass {local-macro-info}

This class represents information about local macros introduced by
\texttt{macrolet}.  An instance of this class is returned by a call to
\texttt{function-info} when it turns out that the function name passed
as an argument refers to a local macro.

\Definitarg {:name}

This initarg supplies the name of the local macro.  This initarg
must be supplied.

\Definitarg {:expander}

This initarg is used to supply the macro function used to expand macro
forms that use this macro.  This initarg must be supplied. 

\Defmethod {name} {(info {\tt local-macro-info})}

Given an instance of the class \texttt{local-macro-info}, this
method returns the name of the local macro as supplied by the
initarg \texttt{:name}.

\Defmethod {expander} {(info {\tt local-macro-info})}

Given an instance of the class \texttt{local-macro-info}, this
method returns the expander of the local macro as supplied by the
initarg \texttt{:expander}.

\subsection{Global macro information}

\Defclass {global-macro-info}

This class represents information about global macros introduced by
\texttt{macrolet}.  An instance of this class is returned by a call to
\texttt{function-info} when it turns out that the function name passed
as an argument refers to a global macro.

\Definitarg {:name}

This initarg supplies the name of the global macro.  This initarg
must be supplied.

\Definitarg {:expander}

This initarg is used to supply the macro function used to expand macro
forms that use this macro.  This initarg must be supplied. 

\Definitarg {:compiler-macro}

This initarg is used to supply a \emph{compiler macro function} when
a compiler macro is associated with the global macro.  If this
initarg is not given, it defaults to \texttt{nil}, meaning that no
compiler macro is associated with this macro. 

\Defmethod {name} {(info {\tt global-macro-info})}

Given an instance of the class \texttt{global-macro-info}, this
method returns the name of the global macro as supplied by the
initarg \texttt{:name}.

\Defmethod {expander} {(info {\tt global-macro-info})}

Given an instance of the class \texttt{global-macro-info}, this
method returns the expander of the global macro as supplied by the
initarg \texttt{:expander}.

\Defmethod compiler-macro {(info {\tt global-macro-info})}

Given an instance of the class \texttt{global-macro-info}, this
method returns the \emph{compiler macro function} information
associated with the global macro, as supplied by the initarg
\texttt{:compiler-macro}.  If that initarg was not supplied, this
method returns \texttt{nil}.

\subsection{Special operator information}

\Defclass {special-operator-info}

This class represents information about special operators.  An
instance of this class is returned by a call to \texttt{function-info}
when it turns out that the function name passed as an argument refers
to a specialoperator.

\Definitarg {:name}

This initarg supplies the name of the special operator.  This initarg
must be supplied.

\Defmethod {name} {(info {\tt special-operator-info})}

Given an instance of the class \texttt{special-operator-info}, this
method returns the name of the special operator as supplied by the
initarg \texttt{:name}.

\section{Block information}

\Defgeneric {block-info} {environment symbol}

\Defcondition {no-block-info}

This condition is signaled by \sysname{} when a client-supplied method
on the generic function \texttt{block-info} returns \texttt{nil}.

\Defmethod {name} {(condition {\tt no-block-info})}

This method returns the name of the block for which no info was
available. 

\Defclass {block-info}

This class represents information about blocks introduced by
\texttt{block}.  An instance of this class is returned by a call to
\texttt{block-info} when the symbol passed as an argument refers to a
block.

\Definitarg {:name}

This initarg supplies the name of the block.  This initarg must be
supplied.

\Definitarg {:identity}

This initarg is used to supply some kind of implementation-defined 
\emph{identity}.  The implementation can supply any object as the
identity, because it is not interpreted by the compiler.  However, the
\emph{same} identity must be supplied each time for a particular
block.  This initarg must be supplied. 

\Defmethod {name} {(info {\tt block-info})}

Given an instance of the class \texttt{block-info}, this method
returns the name of the block as supplied by the initarg
\texttt{:name}.

\Defmethod {identity} {(info {\tt block-info})}

Given an instance of the class \texttt{block-info}, this method
returns the identity of the block as supplied by the initarg
\texttt{:identity}.

\section{Tag information}

\Defgeneric {tag-info} {environment tag}

\Defcondition {no-tag-info}

This condition is signaled by \sysname{} when a client-supplied method
on the generic function \texttt{tag-info} returns \texttt{nil}.

\Defmethod {name} {(condition {\tt no-tag-info})}

This method returns the name of the tag for which no info was
available. 

\Defclass {tag-info}

\Definitarg {:name}

This initarg supplies the name of the tag.  This initarg must be
supplied.

\Definitarg {:identity}

This initarg is used to supply some kind of implementation-defined 
\emph{identity}.  The implementation can supply any object as the
identity, because it is not interpreted by the compiler.  However, the
\emph{same} identity must be supplied each time for a particular
tag.  This initarg must be supplied. 

\Defmethod {name} {(info {\tt tag-info})}

Given an instance of the class \texttt{tag-info}, this method
returns the name of the block as supplied by the initarg
\texttt{:name}.

\Defmethod {identity} {(info {\tt tag-info})}

Given an instance of the class \texttt{tag-info}, this method
returns the identity of the tag as supplied by the initarg
\texttt{:identity}.

\section{Optimization information}

\Defgeneric {optimize-info} {environment}

Client-supplied methods on this function must always return a valid
instance of the class \texttt{optimize-info}. Optimize info includes
information about optimize declarations as well as policies; more
information on policies is in \refChap{chap-policy}.

\Defclass {optimize-info}

\Definitarg {:optimize}

This initarg supplies the information about optimize declarations.
It should be a proper list where each element is a list (quality
value) , i.e. a list valid as the \texttt{cdr} of an \texttt{optimize} 
declaration specifier, with no defaulted values. Such a list can be
returned by \texttt{cleavir-policy:normalize-optimize}. The list
must be complete in the sense that every quality valid in
 the environment (see \texttt{optimize-qualities}) is represented.

This initarg must be supplied.

\Definitarg {:policy}

This initarg supplies the policy for the \texttt{optimize-info}
object. It should be a value returned by
\texttt{cleavir-policy:compute-policy} for the \texttt{:optimize} initarg.

Because the policy is referenced widely during AST generation,
environment query methods should make an effort to cache policies,
and to compute new policy objects only when necessary.

This initarg must be supplied.

\Defmethod {optimize} {(info {\tt optimize-info})}

Given an instance of the class \texttt{optimize-info}, this method 
returns the optimize information as supplied by the initarg
\texttt{:optimize}.

\Defmethod {policy} {(info {\tt optimize-info})}

Given an instance of the class \texttt{optimize-info}, this method returns the policy as supplied by the initarg \texttt{:policy}.

\section{Type information}

\Defgeneric {type-expand} {environment type}

This generic function is called with some arbitrary type specifier,
and the return value is an equivalent type specifier that does not
contain any user-defined types introduced by \texttt{deftype}. 
