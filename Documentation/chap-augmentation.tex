\chapter{Augmenting the environment}

\section{Creating new description}

In order to create a new description, \texttt{make-instance}
must be called, providing values for all the initialization arguments
corresponding to features that do not have any initialization forms.


\section{High-level augmentation functions}
\label{sec-high-level-augmentation-functions}

\subsection{Adding and annotating variables}

\subsubsection{Adding a lexical variable}

{\footnotesize
\Defgeneric {add-lexical-variable} {client environment name \optional identity}
}

This function returns a new environment that is like
\textit{environment} except that it has been augumented with a lexical
variable named \textit{name}.  The optional argument \textit{identity}
can be supplied by client code to distinguish different lexical
variables with the same name.

\subsubsection{Adding a special variable}

{\footnotesize
\Defgeneric {add-special-variable} {client environment name}
}

This function returns a new environment that is like
\textit{environment} except that it has been augumented with a special
variable named \textit{name}.

\subsubsection{Adding a local symbol macro}

{\footnotesize
\Defgeneric {add-local-symbol-macro} {client environment name expansion}
}

This function returns a new environment that is like
\textit{environment} except that it has been augmented with a local
symbol macro named \texttt{name}, with the expansion
\textit{expansion}

\subsubsection{Annotating a variable with a type}
\label{sec-annotating-a-variable-with-a-type}

{\footnotesize
\Defgeneric {add-variable-type} {client environment name type}
}

This function returns a new environment that is like
\textit{environment} except that the variable named \textit{name} has
been annotated with the type specifier \textit{type}.

The type of the variable returned when the new environment is queried
for the variable named \textit{name} will have a new type that is the
conjunction of \textit{type} and the type it had in
\textit{environment}.

This function can be used when \textit{name} names a lexical variable,
a special variable, or a symbol macro, but \emph{not}
when \textit{name} names a constant variable.

\subsubsection{Annotating a variable with an \texttt{ignore} declaration}
\label{sec-annotating-a-variable-with-ignore}

{\footnotesize
\Defgeneric {add-variable-ignore} {client environment name ignore}
}

This function returns a new environment that is like
\textit{environment} except that the variable named \textit{name} has
been annotated with an \texttt{ignore} declaration.

The argument \textit{ignore} must be the symbol \texttt{ignore} or the
symbol \texttt{ignorable}.

This function can be used when \textit{name} names a lexical variable
or a local symbol macro.

\subsubsection{Annotating a variable with a \texttt{dynamic-extent} declaration}
\label{sec-annotating-a-variable-with-dynamic-extent}

{\footnotesize
\Defgeneric {add-variable-dynamic-extent} {client environment name}
}

This function returns a new environment that is like
\textit{environment} except that the variable named \textit{name} has
been annotated with an \texttt{dynamic-extent} declaration.

This function can be used only when \textit{name} names a lexical variable.

\subsection{Adding and annotating functions}

\subsubsection{Adding a local function}

{\footnotesize
\Defgeneric {add-local-function} {client environment name \optional identity}
}

This function returns a new environment that is like
\textit{environment} except that it has been augumented with a local
function named \textit{name}.  The optional argument \textit{identity}
can be supplied by client code to distinguish different functions with
the same name.

\subsubsection{Adding a local macro}

{\footnotesize
\Defgeneric {add-local-macro} {client environment name expander}
}

This function returns a new environment that is like
\textit{environment} except that it has been augmented with a local
macro named \texttt{name}.  The argument \textit{expander} is a
macro-expansion function that takes two arguments, a form and an
environment.

\subsubsection{Annotating a function with a type}
\label{sec-annotating-a-function-with-a-type}

{\footnotesize
\Defgeneric {add-function-type} {client environment name type}
}

This function returns a new environment that is like
\textit{environment} except that the function named \textit{name} has
been annotated with the type specifier \textit{type}.

The type of the function returned when the new environment is queried
for the function named \textit{name} will have a new type that is the
conjunction of \textit{type} and the type it had in
\textit{environment}.

This function can be used when \textit{name} names a local function or
a global function.

\subsubsection{Annotating a function with an \texttt{ignore} declaration}
\label{sec-annotating-a-function-with-ignore}

{\footnotesize
\Defgeneric {add-function-ignore} {client environment name ignore}
}

This function returns a new environment that is like
\textit{environment} except that the function named \textit{name} has
been annotated with an \texttt{ignore} declaration.

The argument \textit{ignore} must be the symbol \texttt{ignore} or the
symbol \texttt{ignorable}.

This function can be used when \textit{name} names a local function or
a local macro.

\subsubsection{Annotating a function with a \texttt{dynamic-extent} declaration}
\label{sec-annotating-a-function-with-dynamic-extent}

{\footnotesize
\Defgeneric {add-function-dynamic-extent} {client environment name}
}

This function returns a new environment that is like
\textit{environment} except that the function named \textit{name} has
been annotated with an \texttt{dynamic-extent} declaration.

This function can be used only when \textit{name} names a local function.

\subsubsection{Annotating a function with an \texttt{inline} declaration}
\label{sec-annotating-a-function-with-inline}

{\footnotesize
\Defgeneric {add-inline} {client environment name inline}
}

This function returns a new environment that is like
\textit{environment} except that the function named \textit{name} has
been annotated with an \texttt{inline} declaration.

The argument \textit{inline} must be the symbol \texttt{inline} or the
symbol \texttt{notinline}.

This function can be used when \textit{name} names a local function or
a local macro.

\subsection{Adding blocks}

{\footnotesize
\Defgeneric {add-block} {client environment name \optional identity}
}

This function returns a new environment that is like
\textit{environment} except that it has been augumented with a block
named \textit{name}, which must be a symbol.  The optional argument
\textit{identity} can be supplied by client code to distinguish
different blocks with the same name.

\subsection{Adding tags}

{\footnotesize
\Defgeneric {add-tag} {client environment tag \optional identity}
}

This function returns a new environment that is like
\textit{environment} except that it has been augumented with a tag
named \textit{tag}, which must be a \emph{go tag}, i.e. a symbol or an
integer.  The optional argument \textit{identity} can be supplied by
client code to distinguish different tags with the same name.

\subsection{Annotating the \texttt{optimize} qualities}

\def\Optimizefunction #1 {This function returns a new environment that
  is like \textit{environment} except that the \texttt{optimize}
  information has been updated with a \texttt{#1} quality value.

The argument \textit{value} must be an integer between $0$ and $3$.}

\subsubsection{Annotating \texttt{optimize} with a \texttt{speed} value}
\label{sec-annotating-optimize-with-speed}

{\footnotesize
\Defgeneric {add-speed} {client environment value}
}

\Optimizefunction{speed}

\subsubsection{Annotating \texttt{optimize} with a \texttt{compilation-speed} value}

{\footnotesize
\Defgeneric {add-compilation-speed} {client environment value}
}

\Optimizefunction{compilation-speed}

\subsubsection{Annotating \texttt{optimize} with a \texttt{debug} value}

{\footnotesize
\Defgeneric {add-debug} {client environment value}
}

\Optimizefunction{debug}

\subsubsection{Annotating \texttt{optimize} with a \texttt{safety} value}

{\footnotesize
\Defgeneric {add-safety} {client environment value}
}

\Optimizefunction{safety}

\subsubsection{Annotating \texttt{optimize} with a \texttt{space} value}

{\footnotesize
\Defgeneric {add-space} {client environment value}
}

\Optimizefunction{space}
