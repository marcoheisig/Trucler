\chapter{Customizing with existing lexical environments}

A typical existing \commonlisp{} implementation has its own
representation of lexical environments and no explicit representation
of the global environment.  In this chapter, we describe the kind of
customization that such an implementation needs to provide in order to
use \sysname{}.

\section{Representing the global environment}
\label{sec-representing-the-global-environment}

Despite the fact that a typical existing implementation has no
first-class object representing the global environment, in order to
customize \sysname{}, the implementation should nevertheless define a
standard class representing such a hypothetical first-class
environment.  In instance of this environment object must be passed to
the language processor, so that when the language processor queries
the null lexical environment for some information, this instance is
passed to the query functions.

\section{Customizing the query functions}

The following query functions are subject to customization:

\begin{itemize}
\item \texttt{describe-variable}
\item \texttt{describe-function}
\item \texttt{describe-block}
\item \texttt{describe-tag}
\item \texttt{describe-class}
\item \texttt{describe-optimize}
\end{itemize}

These functions are described in \refSec{sec-querying-query-functions}.

Only those functions that are actually called by the language
processor need be customized.

The customization consists of supplying methods on the relevant query
functions, specialized to:

\begin{itemize}
\item the specific client class chosen by the implementation, and
\item the classes representing environments in the implementation.
\end{itemize}

Typically, two methods must be supplied, one specialized to the
\emph{lexical} environment class of the implementation, and another
one, specialized to the \emph{global} environment class, as describe
in \refSec{sec-representing-the-global-environment}.  These methods
should return instances of the instantiable classes described in
\refSec{sec-instantiable-query-classes}.

\section{Customizing the augmentation functions}

For an existing implementation, the easiest way to customize
environment augmentation, is to target only the high-level
augmentation functions described in
\refSec{sec-high-level-annotation-functions}.
