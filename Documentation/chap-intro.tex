\chapter{Introduction}
\pagenumbering{arabic}%

In section 8.5 of the second edition of the book ``Common Lisp, the
Language'' (also known as CLtL2) by Guy Steele
\cite{Steele:1990:CLL:95411}, a protocol for accessing compile-time
environments is defined.  That protocol has two main problems:

\begin{enumerate}
\item It is incomplete.  It does not provide for a way to query the
  invironment for description about blocks or tags.
\item It is not extensible.  In order for an implementation to make
  one of the query functions return more description, additional
  return values would have to be defined.  However, such a change is
  not considered backward compatible, so this kind of extension is not
  recommended.
\end{enumerate}

\sysname{} introduces a protocol that solves these problems as
follows:

\begin{enumerate}
\item It contains additional query and augmentation functions for
  blocks and tags.
\item Instead of returning multiple values, the query functions return
  standard objects.  Accessors specialized to the classes of those
  objects provide the information that the protocol in CLtL2
  provides as multiple values.
\end{enumerate}

In addition to providing a mechanism that solves the problems of
protocol presented in CLtL2, we also add several new features that
a compiler must use to process source code.
