\chapter{The reference implementation}

\section{System and package}

The \asdf{} system name for the reference implementation is
\texttt{trucler-reference} and the package name is
\texttt{trucler-reference} as well.

\section{Client and environment}

The reference implementation defines a client class, an instance of
which is to be used to pass as the corresponding \texttt{client}
argument to protocol functions and that class is named
\texttt{client}.

Similarly, the reference implementation defines an environment class that
is used and created by the augmentation methods, and that class is
named \texttt{environment}.

\section{Methods on high-level augmentation functions}

\subsection{Adding and annotating variables}

\subsubsection{Adding a lexical variable}


{\footnotesize
\Defmethod{add-lexical-variable}
{client
 (environment environment)
 name
 \optional identity}
}

This is the main method on \texttt{add-lexical-variable}.  It
instantiates the class \texttt{lexical-variable-description} and then
creates a new environment by calling the function
\texttt{augment-with-variable-description}.

\subsubsection{Adding a special variable}

{\footnotesize
\Defmethod{add-special-variable}
{client
 (environment environment)
 name}
}

This is the main method on \texttt{add-special-variable}.  It
instantiates the class \texttt{special-variable-description} and then
creates a new environment by calling the function
\texttt{augment-with-variable-description}.

\subsubsection{Adding a local symbol macro}

{\footnotesize
\Defmethod{add-local-symbol-macro}
{client
 (environment environment)
 name
 expansion}
}

This is the main method on \texttt{add-local-symbol-macro}.  It
instantiates the class \texttt{local-symbol-macro-description} and then
creates a new environment by calling the function
\texttt{augment-with-variable-description}.

\subsubsection{Annotating a variable with a type}

{\footnotesize
\Defmethod{add-variable-type}
{client
 (environment environment)
 name
 type}
}

This is the main method on \texttt{add-variable-type}.  It
calls \texttt{describe-variable} to obtain an existing variable
description.  It then calls \texttt{merge-type} to create a new
variable description.  Finally, it calls
\texttt{augment-with-variable-description} in order to create and
return a new environment.

\subsubsection{Annotating a variable with an \texttt{ignore} declaration}

{\footnotesize
\Defmethod{add-variable-ignore}
{client
 (environment environment)
 name
 ignore}
}

This is the main method on \texttt{add-variable-ignore}.  It
calls \texttt{describe-variable} to obtain an existing variable
description.  It then calls \texttt{merge-ignore} to create a new
variable description.  Finally, it calls
\texttt{augment-with-variable-description} in order to create and
return a new environment.

\subsubsection{Annotating a variable with a \texttt{dynamic-extent} declaration}

{\footnotesize
\Defmethod{add-variable-dynamic-extent}
{client
 (environment environment)
 name}
}

This is the main method on \texttt{add-variable-dynamic-extent}.  It
calls \texttt{describe-variable} to obtain an existing variable
description.  It then calls \texttt{merge-dynamic-extent} to create a new
variable description.  Finally, it calls
\texttt{augment-with-variable-description} in order to create and
return a new environment.

\subsection{Adding and annotating functions}

\subsubsection{Adding a local function}

{\footnotesize
\Defmethod{add-local-function}
{client
 (environment environment)
 name
 \optional identity}
}

This is the main method on \texttt{add-local-function}.  It
instantiates the class \texttt{local-function-description} and then
creates a new environment by calling the function
\texttt{augment-with-function-description}.

\subsubsection{Adding a local macro}

{\footnotesize
\Defmethod {add-local-macro} {client (environment environment) name expander}
}

This is the main method on \texttt{add-local-macro}.  It instantiates
the class named \texttt{local-macro-description} and then creates a
new environment by calling the function
\texttt{augment-with-function-description}.

\subsubsection{Annotating a function with a type}

{\footnotesize
\Defmethod{add-function-type}
{client
 (environment environment)
 name
 type}
}

This is the main method on \texttt{add-function-type}.  It
calls \texttt{describe-function} to obtain an existing function
description.  It then calls \texttt{merge-type} to create a new
function description.  Finally, it calls
\texttt{augment-with-function-description} in order to create and
return a new environment.

\subsubsection{Annotating a function with an \texttt{ignore} declaration}

{\footnotesize
\Defmethod{add-function-ignore}
{client
 (environment environment)
 name
 ignore}
}

This is the main method on \texttt{add-function-ignore}.  It calls
\texttt{describe-function} to obtain an existing function description.
It then calls \texttt{merge-ignore} to create a new function
description.  Finally, it calls
\texttt{augment-with-function-description} in order to create and
return a new environment.

\subsubsection{Annotating a function with a \texttt{dynamic-extent} declaration}

{\footnotesize
\Defmethod{add-function-dynamic-extent}
{client
 (environment environment)
 name}
}

This is the main method on \texttt{add-function-dynamic-extent}.  It
calls \texttt{describe-function} to obtain an existing variable
description.  It then calls \texttt{merge-dynamic-extent} to create a new
variable description.  Finally, it calls
\texttt{augment-with-function-description} in order to create and
return a new environment.

\subsubsection{Annotating a function with an \texttt{inline} declaration}

{\footnotesize
\Defmethod{add-inline}
{client
 (environment environment)
 name
 inline}
}

This is the main method on \texttt{add-inline}.  It calls
\texttt{describe-function} to obtain an existing function description.
It then calls \texttt{merge-inline} to create a new function
description.  Finally, it calls
\texttt{augment-with-function-description} in order to create and
return a new environment.

\subsection{Adding blocks}

{\footnotesize
\Defmethod{add-block}
{client
 (environment environment)
 name
 \optional identity}
}

This is the main method on \texttt{add-block}.  It instantiates the
class \texttt{block-description} and then creates a new
environment by calling the function
\texttt{augment-with-block-description}.

\subsection{Adding tags}

{\footnotesize
\Defmethod{add-tag}
{client
 (environment environment)
 tag
 \optional identity}
}

This is the main method on \texttt{add-tag}.  It instantiates the
class \texttt{tag-description} and then creates a new
environment by calling the function
\texttt{augment-with-tag-description}.

\subsection{Annotating the \texttt{optimize} qualities}

\def\Optimizemethod #1 {This is the main method on
  \texttt{add-#1}.  It calls \texttt{describe-optimize} to obtain
  the existing optimize description.  It then calls
  \texttt{merge-#1} to create a new optimize description.  Finally,
  it calls \texttt{augment-with-optimize-description} in order to
  create and return a new environment.}

\subsubsection{Annotating \texttt{optimize} with a \texttt{speed} value}

{\footnotesize
\Defmethod{add-speed}
{client
 (environment environment)
  value}
}

\Optimizemethod{speed}

\subsubsection{Annotating \texttt{optimize} with a \texttt{compilation-speed} value}

{\footnotesize
\Defmethod{add-compilation-speed}
{client
 (environment environment)
  value}
}

\Optimizemethod{compilation-speed}

\subsubsection{Annotating \texttt{optimize} with a \texttt{debug} value}

{\footnotesize
\Defmethod{add-debug}
{client
 (environment environment)
  value}
}

\Optimizemethod{debug}

\subsubsection{Annotating \texttt{optimize} with a \texttt{safety} value}

{\footnotesize
\Defmethod{add-safety}
{client
 (environment environment)
  value}
}

\Optimizemethod{safety}

\subsubsection{Annotating \texttt{optimize} with a \texttt{space} value}

{\footnotesize
\Defmethod{add-space}
{client
 (environment environment)
  value}
}

\Optimizemethod{space}
